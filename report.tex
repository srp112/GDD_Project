\documentclass[12pt,letterpaper]{article}
\usepackage[latin1]{inputenc}
\usepackage{amsmath}
\usepackage{amsfonts}
\usepackage{amssymb}
\usepackage{graphicx}

\graphicspath{{./plots/}}
\begin{document}
\title{GDD Project \\ \vspace{.5 cm} {\Large Computer Based Course and Application Tools Project} }
\author{Please add your name~Farhad M. Kazemi ~ Please add your name}
\maketitle
\begin{abstract}
The aim of this research is to find the...

The results showed that...
\end{abstract}

\section{Introduction}

In this report, we would like to...

On the rest of this report, firstly we will describe our dataset properties and what we have done to clean this data and to overcome data heterogeneity and missing values. In the second section, we briefly described ....
Finally, we have a conclusion and what could be done in the future works.
\subsection{Dealing With Dataset}
The dataset consists of 

The input dataset has 
\begin{enumerate}
\item \textbf{G:} We used this method for imputing missing data and extracting features on raw data (... sample with features). More details are available in section 2.
\end{enumerate}
\noindent Because of our time limit and for avoiding further complexity, we only considered 3 datasets which are ... 
It means we have datasets with ... samples and ... features respectively. Table shows the datasets and their number of samples.
\section{Methods}
In this section, we will describe principles of methods that we used for analyzing data. We will explain a ... as well as ....  Furthermore, we will describe .... 
\section{Question 3}
Question 3:

Write a command line program that takes arguments

In the GDD calculation part, we have used the package argparse to handle the command line arguments. First, we set the csv argument which completes file path of cities file. Second, we set the tbase argument(default = 10) and tupper argument(default = 30) to apply in the GDD calculation. 

The data needed for GDD calculation are already downloaded and stored in the ‘data’ folder with their filenames specified in ‘citiex.txt’ file. In this program, it read these data directly from ‘data’ folder, do calculation on each file, and store them separately. For the missing information in csv file, we will set the GDD to 0. The GDD calculation equation is below:


In our program, we set the  value to the minimum between tupper and the maximum daily temperature and  to the 10. For example, on 15th, August, 2010 in Vancouver, the maximum temperature is 31.2 Celsius degrees which is more than 30 Celsius degrees, and the minimum temperature is 16.1 Celsius degrees. Hence the GDD equation to (30+16.1)/2 – 10, and the GDD result is 13.05.

Finally, we add the new ‘GDD’ column to the existing data files, which is supposed to prevent creating more files in the project.

In the next part, we are supposed to use the GDD calculation to create plots showing accumulated GDD vs time for selected cities.
\subsection{ section}

\section{Question 3}
Question 3:

Write a command line program that takes arguments

In the GDD calculation part, we have used the package argparse to handle the command line arguments. First, we set the csv argument which completes file path of cities file. Second, we set the tbase argument(default = 10) and tupper argument(default = 30) to apply in the GDD calculation. 

The data needed for GDD calculation are already downloaded and stored in the ‘data’ folder with their filenames specified in ‘citiex.txt’ file. In this program, it read these data directly from ‘data’ folder, do calculation on each file, and store them separately. For the missing information in csv file, we will set the GDD to 0. The GDD calculation equation is below:


In our program, we set the  value to the minimum between tupper and the maximum daily temperature and  to the 10. For example, on 15th, August, 2010 in Vancouver, the maximum temperature is 31.2 Celsius degrees which is more than 30 Celsius degrees, and the minimum temperature is 16.1 Celsius degrees. Hence the GDD equation to (30+16.1)/2 – 10, and the GDD result is 13.05.

Finally, we add the new ‘GDD’ column to the existing data files, which is supposed to prevent creating more files in the project.

In the next part, we are supposed to use the GDD calculation to create plots showing accumulated GDD vs time for selected cities.
\begin{figure}
\centering
\includegraphics[scale=0.6]{GDDMap_2015.png}
\caption{F}
\end{figure}

\begin{figure}
\centering
\includegraphics[scale=0.6]{toronto,ontario_GddPlot.png}
\caption{B}
\label{fig:gbm}
\end{figure}

\begin{figure}
\centering
\includegraphics[scale=0.6]{{"st. john's,newfoundland_GddPlot"}.png}
\caption{A}
\label{fig:gbm}
\end{figure}

\begin{figure}
\centering
\includegraphics[scale=0.6]{{"vancouver,british columbia_GddPlot"}.png}
\caption{C}
\end{figure}

\begin{figure}
\centering
\includegraphics[scale=0.6]{{"st. john's,newfoundland_MinMax"}.png}
\caption{D}
\end{figure}

\begin{figure}
\centering
\includegraphics[scale=0.6]{toronto,ontario_MinMax.png}
\caption{E}
\end{figure}

\begin{figure}
\centering
\includegraphics[scale=0.6]{{"vancouver,british columbia_MinMax"}.png}
\caption{F}
\end{figure}


%\includegraphics{plot00.png}
%\includegraphics{Plot01.png}
%\includegraphics{Plot02.png}
%\includegraphics{Plot03.png}
%\includegraphics{Plot04.png}
%\includegraphics{Plot05.png}
\subsubsection{section}
\section{Experimental Results}
In our experiments, we used 
\subsection{result}
Parameter settings listed in table 1 was used for exploring the performance of 
\section{Discussion}
\section{Conclusion}
In this work ...
\newpage
\end{document}
